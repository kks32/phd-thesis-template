%!TEX root = ../thesis.tex
%*******************************************************************************
%****************************** Second Chapter *********************************
%*******************************************************************************

\chapter{The SNO+ Detector}\label{chap:detector}
\epigraph{\textit{The light-soaked days are coming.}}{\textsc{John Green}}
\section{Detector Geometry and Design}
\nomenclature{\textbold{SNO}}{Sudbury Neutrino Observatory}
The SNO+ detector is a large, multi-purpose neutrino detector built in the SNOLAB underground laboratory near Sudbury, Canada. Its main detector structure is taken from the Nobel prize-winning Sudbury Neutrino Observatory (SNO)~\cite{}, % cite nobel prize for Art
which can be seen in Fig.~\ref{fig:snoplus_detector}. The bulk of the detector is the main detector medium, which changes depending on the phase of the experiment --- more on the specifics of this shortly. This medium is held within a \SI{12}{\metre} diameter sphere known as the Acrylic Vessel (AV). The AV floats within a body of ultra-pure water (UPW), beyond which is a stainless steel support structure (PSUP) that holds $\sim\num{9000}$ Photomultiplier Tubes (PMTs). It is these PMTs that detect the light generated from physics events that occur within the detector medium. The AV is kept in place relative to the PSUP through a series of `hold-up' and `hold-down' tensylon ropes. All of these components are suspended within a large cylindrical cavity also filled with UPW. Directly above the detector is the Deck, within which all the detector electronics are kept. Access within the AV for calibration tools and filling is possible only through the acrylic `neck' on top of the AV.

\begin{figure}
    \centering
    \includegraphics[width=0.48\linewidth]{2_Detector/Figs/detector_picture.png}
    \caption[3D model of the SNO+ detector]{3D model of the SNO+ detector~\cite{albaneseSNOExperiment2021}.}
    \label{fig:snoplus_detector}
\end{figure}

This choice of design is highly deliberate, with the details are discussed in~\cite{albaneseSNOExperiment2021}. 
The location \SI{2.2}{\km} underground ensures that there is minimal impact from cosmic rays: only 3 cosmic ray muon events are expected within the detector an hour~\cite{}. % cite something - Lorna's thesis?
By making the detector spherical, the high degree of symmetry can easily be taken advantage of in event reconstruction and analysis. Moreover, light produced throughout most of the body of the AV will be minimally-impacted by refraction through the acrylic. The only major exception to this is light emitted within $\sim\SI{50}{\cm}$ of the AV, at which point total internal reflection becomes possible. In order to make as much emitted light be able to get detected as possible, all materials within the PSUP were chosen for their optical transparency (excepting the ropes).

Another major design consideration is that of radioactive backgrounds. Maintaining minimal levels of backgrounds is critical for effective particle physics, otherwise the signal one is searching for would become completely swamped. Materials within the detector were chosen to ensure these background levels would be low enough for the Collaboration's physics goals to be achieved. Another benefit of the spherical design is its high volume-to-surface area ratio, which means that the relatively-high background levels of the PMT glass are kept far away from the detection medium.

\section{Experimental Phases}
As mentioned earlier, SNO+ was designed to fulfil a number of physics goals over multiple `phases' of the detector's lifetime. The phases are distinguished by the medium that fills the AV. The first main phase (after a brief `air-filled' phase used only for detector commissioning) was that of the `water-fill', with data taken between May 2017 and July 2019. This was used to perform fundamental optical calibrations of the detector~\cite{}, % Water optics paper
measurements of the solar neutrino flux~\cite{}, % water solar papers
observation of neutrino oscillations in reactor anti-neutrinos~\cite{}, % water antinu paper
and searches for nucleon decay~\cite{}. %nucleon decay papers

After this, the detector was filled with 800 tonnes of liquid scintillator known as linear alkylbenzene (LAB), mixed with the fluor PPO. More information on the physics of scintillators can be found in Section~\ref{sec:interactions_w_matter}. Filling of the LABPPO cocktail had to be paused in March 2020 due to the COVID-19 pandemic, leading to the detector having its bottom half still filled with UPW, and the top half filled with LAB and PPO at \SI{0.5}{\gpl}. This impromptu phase became known as the `partial-fill', and allowed for some creative analyses to be performed: an initial neutrino oscillation analysis from reactor anti-neutrinos~\cite{}, % Iwan's thesis & forthcoming paper
as well as the first ever observation of directionality in a high light yield scintillator~\cite{}. % Josie's thesis & forthcoming PRL
Eventually, filling of the detector with liquid scintillator was able to resume, being completed in May 2021. At that point, the concentration of PPO in the detector was at \SI{0.6}{\gpl}, markedly below the target level of \SI{2.0}{\gpl}. A further `PPO top-up' campaign then proceeded, finishing in April 2022 with a final concentration of \SI{2.2}{\gpl} PPO. Thus began the `scintillator-fill' of the experiment, which continues on during the time of writing. The main goals for this phase include a number of solar neutrino analyses (including the one described in Chapter~\ref{chap:solar_osc_analysis}), a precision measurement of the neutrino oscillation parameter \dmsq{} using reactor anti-neutrinos~\cite{}, % Iwan's thesis
and further calibrations of the detector and its backgrounds.

Finally, in the near future the detector will be loaded with Tellurium, allowing for the flagship analysis of the experiment to begin: neutrinoless double beta decay. The details of this chemical loading process are described in~\cite{}. % Te loading paper
Alongside the Te-loaded liquid scintillator will be a number of other chemicals within the scintillator cocktail that will help ensure optimal optical properties and chemical stability. These include: the surfactant DDA to ensure the solubility and stability of the `Te-diol' within the LAB; the wavelength-shifter BisMSB to absorb light at short wavelengths and re-emit closer to the optimal quantum efficiency of the detector's PMTs; and the anti-oxidant BHT to prevent any free-radicals within the liquid scintillator from `yellowing' the medium.

% \begin{itemize}
%     \item Describe the SNO+ geometry at a high level: explain structure, and why certain design choices were made.
%     \item Describe standard coordinate axis; note AV offset.
%     \item Mention the main phases of SNO+, both past, present, and future.
% \end{itemize}
\section{Detecting and Recording an Event in SNO+: A Journey}\label{sec:event_journey}
To understand well the SNO+ detector, it is worth thinking about how the information of a physics event, e.g. a solar neutrino interaction, gets observed. This section follows the journey of such an event.
\subsection{Particle Interactions with Matter}\label{sec:interactions_w_matter}
All observable physics events within the detector begin by the generation of some form of ionising radiation: $\alpha$, $\beta^{\pm}$, $\gamma$, $p$ or $n$. These can be created via numerous processes, both exciting (e.g. \onbb{} or interactions of neutrinos) and banal (e.g. decay of background radioisotopes): see Section~\ref{sec:background_processes} for some of them. Regardless of their origin, these particles begin propagating through the detector, and interacting with the detector medium. A number of mechanisms then allow for the generation of optical-wavelength light as a result of these interactions.
\subsubsection{Cherenkov Light Emission}
Whenever a charged particle passes through a dielectric medium, nearby molecules polarise and attempt to align their dipoles in the direction of the charge. This results in a temporary polarisation of the medium as the charge passes though it. However, the molecules can only respond to the charge's movement at the speed of light in the medium, which is necessarily slower than the speed of light in a vacuum by a factor of the medium's refractive index. If the charged particle is able to move at a speed greater than the speed of light in the medium, then the dipoles formed within the medium struggle to respond to the motion. This superluminal motion results in a wake of polarisation in the medium, which propagates outwards from the direction of the charged particle: this is Cherenkov light. This process is much akin to the `sonic boom' that occurs when an object travels at supersonic speeds.

Cherenkov light emanates outwards in a cone along the direction of the charge's travel; the angle of the cone $\theta_{\gamma}$ is purely a function of the speed of the charged particle relative to the speed of light, $\beta$, and the refractive index of the medium $n(\lambda)$: $\cos{\theta_{\gamma}}(\lambda) = \frac{1}{n(\lambda)\beta}$. This formula also demonstrates that there is a minimum speed necessary for Cherenkov light to be generated: $\beta = 1/n(\lambda)$.

All detection media are capable of allowing for Cherenkov light to be generated, as long as sufficiently high energy particles can be produced. In the water-fill phase of the detector, Cherenkov light was the only means by which light could be generated. Light from Cherenkov emission can still be created in liquid scintillator, but it tends to be swamped by another form of light generation: scintillation.
\subsubsection{Scintillation}\label{sec:scintillation}
Whenever a high-energy particle passes through a medium, interactions between the particle and the medium's molecules lead to an inexorable transfer of energy from the former to the latter. These interactions are numerous, and are a function of the type of particle and its kinetic energy. For example, a \SI{5}{\MeV} electron loses energy from electromagnetic interactions with nearby molecules as it passes through, and scatters after colliding with atomic electrons or nuclei. From the perspective of the medium, its molecules use this new-found energy for a number of activities:
\begin{enumerate}
    \item Exciting atomic electrons into various higher-energy states;
    \item Ionising atomic electrons;
    \item Increasing the overall kinetic energy of the molecule.\label{enum:heat}
\end{enumerate}
At the macroscopic level, process~\ref{enum:heat} is simply the conversion of the high energy particle's energy into heating the medium. In fact, within most materials the other processes also end up doing the same.

However, for certain special classes of material the excitation and ionisation of electrons can lead to a unique process: scintillation (often generally referred to as `luminescence' or `fluorescence'). Here we choose to focus on the particular case of organic liquid scintillators, given that this is the type of scintillator which currently fills the SNO+ detector. Certain organic compounds (i.e. molecules built from a skeleton of carbon atoms) have `$\pi$-bonds' in addition to the single `$\sigma$-bond' that is needed to bond two carbon atoms together. A major example of these $\pi$-bonds is in the benzene rings of `aromatic' hydrocarbons, such as LAB and PPO. Fig.~\ref{fig:benzene_pi_bonds} shows an image of the orbitals for these $\pi$-bonds in a benzene ring: note how the electrons that inhabit these particular orbitals become entirely delocalised.

\begin{figure}
    \centering
    % \includegraphics[]{}
    \caption[]{}
    \label{fig:benzene_pi_bonds}
\end{figure}

Because of this delocalised structure, excited atomic $\pi$-electrons can stay in what is typically the first-excited state for somewhat longer than typical excited states: lifetimes of $\mathcal{O}(\SI{e-9}{\second})$ as opposed to $\mathcal{O}(\SI{e-12}{\second})$. Moreover, decays from this state can emit light typically in the optical-wavelength range. It is this light emission that is called `scintillation light'. Ionised electrons can also recombine --- that is, re-enter atomic orbitals --- into various excited states, and then decay back to the ground state, also allowing for the possibility of scintillation light to be generated.

The nature of atomic spin selection rules restrict singlet ground-state electrons from exciting into triplet orbitals. However, electrons that were ionised and then recombined have no such restriction, and so are able to enter various excited triplet states. The same selection rules that restrict excitation from singlet into triplet states also works in the other direction, restricting the ability of electrons in excited triplet states from decaying back into the ground state. This does eventually happen, but the lifetimes of these ``phosphorescence'' decays are typically order of magnitude slower than their singlet-singlet counterparts. As a result, scintillation light typically has, to first order, a `fast' and `slow' time component, which can be seen in Fig.~\ref{fig:typical_tres_dist_physics}.\footnote{In SNO+, we currently model emission of scintillation light from LAB with 3 time components, each roughly one order of magnitude slower than the other.}

\begin{figure}
    \centering
    % \includegraphics[]{}
    \caption[]{}
    \label{fig:typical_tres_dist_physics}
\end{figure}

When using just a single scintillating compound, the very same energy levels that can generate scintillation light are those that can absorb it. This can be a problem for large-scale detectors like SNO+, which depend on scintillation light being unobstructed in its path to the PMTs. Conveniently, this problem can be addressed with the addition of another scintillating component, known (somewhat confusingly) as the primary fluor. In SNO+, this is the PPO added to the LAB.

When an LAB molecule is excited, that energy can be transferred to a PPO molecule through what is known as a `non-radiative transfer'. In short, this transfer of energy occurs not through the emission and absorption of optical photons, but through the coupling of the molecules' electric dipoles.\footnote{To be pedantic, photons are still transferred in this energy exchange, but they are virtual instead of real.}
When the now-excited PPO molecule de-excites to emit scintillation light, the different molecular structure it has generates a different emission spectrum to that of LAB. These longer wavelengths of light are no longer able to be absorbed by the LAB, allowing for a scintillator with less optical absorption.

Adding in one additional component doesn't have to be the end, either. In SNO+ we plan on adding in the compound BisMSB to the scintillator cocktail. This is a `wavelength-shifter': scintillation light at short wavelengths is absorbed, and then re-emitted at longer wavelengths, where the detection efficiency of the PMTs is greater. More on the properties of the PMTs in SNO+ can be found in Section~\ref{sec:pmts}. The net effect of the three scintillating components within SNO+ can be seen in Fig.~\ref{fig:scintillator_abs_emit_dist}. Note how, as energy is transferred from one scintillation component to another, the wavelength of light emitted gets necessarily longer as energy is lost to heat.

\begin{figure}
    \centering
    % \includegraphics[]{}
    \caption[]{}
    \label{fig:scintillator_abs_emit_dist}
\end{figure}

The light yield of a scintillator, i.e. the amount of optical photons generated per unit of energy deposited into the scintillator, is a function not just of the scintillator but also the incident particle. In particular, $\alpha$ particles are far more effective at exciting and ionising nearby atoms, and so can deposit far more of its energy into the scintillator per unit volume. However, the strength of this ionisation for $\alpha$s can actually become at detriment to the generation of scintillation light. Empirically, scintillators follow to first order Birks' Law for their scintillation light yield~\cite{}: % Birks Law ref.
\begin{equation}
    \frac{dL}{dx} = S\frac{\frac{dE}{dx}}{1+k_{\mathrm{Birks}}\frac{dE}{dx}},
\end{equation}
where $\frac{dL}{dx}$ is the number of photons emitted per unit track length, $\frac{dE}{dx}$ is the energy loss of the incident particle per unit track length, $S$ is the scintillator's characteristic light yield constant, and $k_{\mathrm{Birks}}$ is the scintillator's ``Birks' Constant''\footnote{Birks' Constant is often just written as $k_{B}$, but this is easily confused with the far-better known Boltzmann Constant, which is completely different!}. 
For minimum-ionising particles such as a \SI{6}{\MeV} electron, the denominator of this equation is close to 1, and so the amount of scintillation light generated is just $\frac{dL}{dx} = S\cdot\frac{dE}{dx}$. However, for $\alpha$-particles generated in radioactive decays, this denominator can become substantial, and in the limiting case we have merely $\frac{dL}{dx} = S$. In the current phase of SNO+, $S$ and $k_{\mathrm{Birks}}$ are measured to be \SI{14000}{\gamma\per\MeV} and \SI{0.0798}{\mm\per\MeV},\footnote{This is the Birks' constant used in simulation for $\alpha$-particles. Other particles, such as electrons, are given slightly different values of $k_{\mathrm{Birks}}$.}
 respectively~\cite{}. % cite where we got S and kb values from.

 \subsection{Optical Processes}
 Once optical-wavelength photons have been created within the detector, various processes can then occur that can hinder its path towards a PMT, and therefore modify the observed signal. This subsection covers the main optical processes, with a focus on Rayleigh scattering, as an understanding of this phenomenon is critical for Chapters~\ref{chap:smellie_hardware}--\ref{chap:smellie_analysis}.
 \subsubsection{Rayleigh Scattering}
Optical scattering is the general process of how light is scattered by particles within a medium. This is fundamentally an electrodynamical process: an electromagnetic wave is incident on the set of particles within the medium, which induces these particles to oscillate within the field, and therefore generating their own electromagnetic radiation in response. Usually, this `scattered' radiation has the same frequency as that of the incident radiation, and therefore the scattering is said to be \textit{elastic}. It is possible under certain circumstances for this scattered radiation to be of a longer wavelength than the incident radiation: in which case, energy was absorbed by the particles and so the scattering was \textit{inelastic}. However, this latter type of scattering, also known as Raman scattering, is not relevant for SNO+~\cite{}. % ref something that allows us to ignore Raman scattering!

The general solution to elastic optical scattering was first described by Gustav Mie~\cite{} % ref Mie theory paper
and Ludvig Lorenz~\cite{} % Lorenz's paper
in what is now known as \textit{Mie Theory}. In this theory, it is assumed that a plane wave of wavelength $\lambda$ is incident on a dielectric sphere of radius $a$. While the general solution to the problem of Mie scattering is somewhat complicated (if tractable), in certain regimes one can make further simplifying assumptions that substantially reduce the complexity of the result. In particular, if one assumes that the size of the particle is much smaller than the wavelength of light, and that any induced dipole moment can actually be established in the time window allowed by the oscillation period of the electromagnetic field~\cite{}, % Rayleigh criteria
then one can obtain \textit{Rayleigh scattering}. This simpler case is so-called because of its initial formulation by Lord Rayleigh~\cite{}. % Rayleigh's scattering paper

One can show that the differential cross-section associated with Rayleigh scattering of unpolarised light off of a single particle, $\frac{d^{2}\sigma_{\textrm{Ray}}}{d\theta d\phi}\left(\theta,\phi\right)$,  is given by~\cite{}: % cite something that gets rayleigh scattering formula
\begin{equation}
    \frac{d^{2}\sigma_{\textrm{Ray}}}{d\theta d\phi}\left(\theta,\phi\right) = \frac{8\pi a^{6}}{\lambda^{4}}\left(\frac{n_{\textrm{par}}^{2}-1}{n_{\textrm{par}}^{2}+2}\right)^{2} \left(1+\cos^{2}\theta\right).
\end{equation}
Here, $\theta$ and $\phi$ correspond respectively to the polar and azimuthal angles of the scattered waves, and $n_{\mathrm{par}}$ is the refractive index of the scattering particle. Most important to notice about this equation is that the cross-section follows a strong $1/\lambda^{4}$ dependence, meaning that short wavelengths of light will be scattered to far greater extents than that of longer wavelengths. Secondly, the light is not scattered isotropically, but according to a $1+\cos^{2}\theta$ dependence. This means that most light is either scattered directly forwards or backwards (known as a \textit{backscattering}), and little gets scattered orthogonally to the direction of the incident light. This is useful when it comes to trying to measure scattering in the SNO+ detector, as it provides a handle upon which to distinguish scattered light from isotropically-emitted scintillation light.

Of course, we care about the scattering that occurs within an entire bulk medium, not just the scattering off of a single molecule. From a macroscopic perspective, the key quantity of interest is a material's \textit{Rayleigh scattering length}, $l_{\textrm{Ray}}$: the mean distance a photon is expected to travel before Rayleigh scattering. One can show that, assuming the above differential scattering cross-section, the Rayleigh scattering length is given by~\cite{}: % https://arxiv.org/pdf/1504.00987.pdf
\begin{equation}
    l_{\textrm{Ray}} = \left[\frac{16\pi}{3}R\right]^{-1}.
\end{equation}
$R$ is the \textit{Rayleigh ratio}, $R=\frac{1}{V}\frac{d^{2}\sigma_{\textrm{Ray}}\left(\ang{90}\right)}{d\theta d\phi}$, equivalent to the power of the scattered light per unit volume of the scattering medium per unit incident intensity at $\theta=\ang{90}$.

This can lead to a few changes to Rayleigh scattering that are worth noting. Firstly, unlike for a single particle, the electric polarisability of a material can be \textit{anisotropic}. Anisotropic materials have a modified angular dependence on their differential cross-section, governed by the \textit{depolarisation ratio}, $\delta$. In particular, the $\left(1+\cos^{2}\theta\right)$ dependence becomes $\left(1+\frac{1-\delta}{1+\delta}\cos^{2}\theta\right)$. For isotropic materials, $\delta=0$, and so the angular dependence reduces to the original form.

Secondly, the above model has been shown to be insufficient to describe liquids or solids~\cite{}, % optical aspects of oceanography!
because of the non-negligible strength of their inter-molecular forces. Fortunately, Einstein~\cite{}, %
Smoluchowski~\cite{}, %
and Cabannes~\cite{} %
developed a theory for describing how photons can scatter off of the local charge density fluctuations that naturally are present in a medium because of the thermal motion of molecules. The theory shows that the Rayleigh ratio of a medium is related to the medium's dielectric constant, $\varepsilon$, by:
\begin{equation}
    R = \frac{\pi^{2}}{2\lambda^{4}}\left[\rho\left(\frac{\partial\varepsilon}{\partial\rho}\right)_{T}\right]^{2} k_{B}T \kappa_{T}\frac{6+6\delta}{6-7\delta},
\end{equation}
where $\rho$ is the density of the medium, $\left(\frac{\partial\varepsilon}{\partial\rho}\right)_{T}$ is the partial derivative of the dielectric constant with respect to a changing density assuming a constant temperature $T$, $k_{B}$ is the Boltzmann Constant, and $\kappa_{T}$ is the medium's isothermal compressibility. This latter quantity is given by the rate of change of volume given a changing pressure of the medium, all at a constant temperature.

Furthermore, the Eykman Equation~\cite{} % I LIKE EYK!
has been shown to be an effective empirical formula relating how $\varepsilon$ is impacted by density fluctuations to the medium's refractive index, $n_{\textrm{med}}$:
\begin{equation}
    \rho\left(\frac{\partial\varepsilon}{\partial\rho}\right)_{T} = 
    \frac{\left(n_{\textrm{med}}^{2}-1\right)\left(2n_{\textrm{med}}^{2}+0.8n_{\textrm{med}}\right)}{n_{\textrm{med}}^{2}+0.8n_{\textrm{med}}+1}.
\end{equation}
This leads to a final formula for the Rayleigh scattering length:
\begin{equation}
    l_{\mathrm{Ray}} = \left[
        \frac{8\pi^{3}}{3\lambda^{4}}
        \left(
            \frac{\left(n_{\textrm{med}}^{2}-1\right)\left(2n_{\textrm{med}}^{2}+0.8n_{\textrm{med}}\right)}{n_{\textrm{med}}^{2}+0.8n_{\textrm{med}}+1}
        \right)^2
        k_{B}T \kappa_{T}\frac{6+3\delta}{6-7\delta}
        \right]^{-1}.
\end{equation}

Discussions of the scattering lengths currently assumed within SNO+'s optical model for UPW and LABPPO can be found within the theses of Krishanu Majumdar~\cite{majumdarMeasurementOpticalScattering2015} and Esther Turner~\cite{turnerMeasurementScatteringCharacteristics2022}. In particular, whilst the scattering length of the UPW in the water phase was measured by Esther, major systematics in the measurement remained. Measurements of the scattering lengths in scintillator are the focus of Chapters~\ref{chap:smellie_hardware}--\ref{chap:smellie_analysis}.



    % \begin{itemize}
    %     \item Explain the electrodynamical model for scattering, giving rise to the expected $1/\lambda^4$ wavelength-dependence of the scattering length, as well as the $1+\cos^2\theta$ dependence of the scattering angle.
    %     \item Describe how the density-fluctuation theory gives rise to a possible modification to the scattering angle distribution due to the anisotropy of the optical media's polarisability vector. Studies by JUNO have shown that LABPPO has such a measurable anisotropy.
    %     \item Show the existing model for water and LABPPO used in RAT, and note that this anisotropy is currently not included in any simulations. I don't need to go in too much detail for this subsection as Krish did a nice job in his thesis and I can cite that, but I do need to write enough to cover the basics for my SMELLIE analysis chapter.
    % \end{itemize}
    % [3 pages]

\subsubsection{Absorption and Re-emission}
In addition to scattering, an optical medium is also able to absorb light that propagates through it. For a given medium, the \textit{absorption length} $l_{\mathrm{abs}}$ is analogous to $l_{\mathrm{Ray}}$ described above, and is typically strongly a function of wavelength. For most materials, absorbed light is forever lost, converted into heat. However, for the special case of scintillators, re-emission of absorbed light is possible: this is because of the physics described in Section~\ref{sec:scintillation}.

Because both scattering and absorption impede a photon's ability to propagate through a medium directly, it is often possible to measure their combined impact through what is known as the absorption/extinction length, $l_{\mathrm{ext}}$:
\begin{equation}\label{eq:ext_length_def}
    \frac{1}{l_{\mathrm{ext}}} = \frac{1}{l_{\mathrm{abs}}} + \frac{1}{l_{\mathrm{Ray}}}.
\end{equation}
In the water phase, the `Laserball' calibration system was used to measure various optical properties of the detector, including the extinction lengths of the UPW and acrylic as a function of wavelength~\cite{andersonOpticalCalibrationSNO2021}. Using the water phase scattering measurements made by Esther, Eq.~\ref{eq:ext_length_def} allowed for the estimation of the absorption lengths of these two materials, shown in Figure~\ref{fig:abs_lengths_optics_paper}. Measurements of the extinction length in the scintillator phase is discussed in detail in Chapter~\ref{chap:smellie_analysis}.

\begin{figure}
    \centering
    \begin{subfigure}{0.98\textwidth}
        \centering
        \includegraphics[width=0.9\textwidth]{2_Detector/Figs/WaterAbsorption.png}
        \caption{UPW optical absorption}
        \label{fig:abs_length_water_optics_paper}
    \end{subfigure}
    \begin{subfigure}{0.98\textwidth}
        \centering
        \includegraphics[width=0.85\textwidth]{2_Detector/Figs/AcrylicAttenuation.png}
        \caption{Acrylic optical attenuation}
        \label{fig:abs_length_acylic_optics_paper}
    \end{subfigure}
    \caption[Measured properties of the UPW and acrylic in the water phase.]{Measured properties of the UPW and acrylic in the water phase, from~\cite{andersonOpticalCalibrationSNO2021}.}
    \label{fig:abs_lengths_optics_paper}
\end{figure}

    % \begin{itemize}
    %     \item State that light can get absorbed by materials, and if that medium is a scintillator then re-emission is possible. I think further details such as the specific shape of the absorption/re-emission of the scintillator and water can be shown in the SMELLIE analysis chapter, in which I have to explain about possible changes to the model anyway.
    % \end{itemize}
    % [1/2 page]
\subsubsection{Surface reflection and refraction}
When light travels through the boundary of one medium to another, both reflection and refraction can be possible, depending on the relative refractive indices of the two media. The refractive indices of the UPW, acrylic, and LABPPO are shown as a function of wavelength in Figure~\ref{fig:ref_indices_snoplus}. Note that, for most optical wavelengths, LABPPO has a very close refractive index to acrylic, whereas UPW is somewhat farther away. By consequence, negligible refraction is expected in most cases for light travelling between the liquid scintillator and the acrylic; however, substantial refraction is possible for light travelling between acrylic and UPW. Because of this, isotropically-emitting point-like physics events within the AV that are close enough to the acrylic will have some of their light undergo Total Internal Reflection (TIR) at the AV, reflecting back into the AV instead of continuing outward into the outer water.

\begin{figure}
    \centering
    % \includegraphics[]{}
    \caption[]{}
    \label{fig:ref_indices_snoplus}
\end{figure}

Even when not undergoing TIR, some light at a boundary can still reflect. The fraction of light that reflects is known as the \textit{reflectance} $R$, compared to that which is able to transmit through the boundary, the \textit{transmittance} $T=1-R$. The \textit{Fresnel Equations} determine the reflectance of an interface~\cite{}:% Fresnel eqs: Hecht?
\begin{equation}
    R_{s} = \left|\frac{n_{1}\cos{\theta_{i}}-n_{2}\cos{\theta_{t}}}{n_{1}\cos{\theta_{i}}+n_{2}\cos{\theta_{t}}}\right|^{2},\\
    R_{p} = \left|\frac{n_{1}\cos{\theta_{t}}-n_{2}\cos{\theta_{i}}}{n_{1}\cos{\theta_{t}}+n_{2}\cos{\theta_{i}}}\right|^{2},
\end{equation}
where $R_{s}$ and $R_{p}$ are the reflectances of $s$- and $p$-polarised light, $n_{1}$ and $n_{2}$ are the refractive indices of the first and second optical media, and $\theta_{i}$ and $\theta_{t}$ are the angles of incidence and refraction, respectively. For SNO+, we are only interested in unpolarised light, so the total reflectance $R = \left(R_{s}+R_{p}\right)/2$.

    % \begin{itemize}
    %     \item State that boundaries between media can induce reflections and refraction, as governed by the Fresnel transmission and reflection formulae. These formulae are worth mentioning because I use them in my SMELLIE extinction length analysis.
    % \end{itemize}
    % [1/2 page]
    \subsection{Detection by PMTs}\label{sec:pmts}
{
    \color{blue}
    \begin{itemize}
        \item Light gets detected via the PMTs. Note the existence of the PMT concentrators to maximise coverage within the AV, but minimise it in the external water: show the calibrated angular response.
        \item Remind reader that a PMT converts photons to photoelectrons with a certain quantum efficiency, dependent on the wavelength of light.
        \item The process of multiplying the signal induces a spread in the possible generated time of the voltage signal, known as the PMT's transit time spread.
        \item The PMT response is also weakly dependent on the number of photoelectrons generated; not enough to be able to confidently distinguish the npe under most circumstances.
    \end{itemize}
    [2 pages]
    \subsection{Data Acquisition and Triggering}
    \begin{itemize}
        \item Summarise how a PMT signal becomes digitised via the front-end electronics, briefly.
        \item Summarise the triggering system, as I'll have to describe in a later chapter how the SMELLIE hardware fits into this, especially as there have been ongoing SMELLIE triggering issues worth mentioning there!
        \item Note the information stored by an event: importantly for this thesis, the TAC and QHS per hit, the event's GTID as well as trigger time measured from the \SI{50}{\mega\Hz} clock. The latter is worth mentioning as this is how I determine time differences for my BiPo tagging in the solar analysis.
        \item Finally, note that these get written to file in the ZDAB format.
    \end{itemize}
    [3 pages]
    \subsection{Operation of the Detector}
    \begin{itemize}
        \item Detector electronics operated through ORCA; allows for different running modes, such as calibrations.
        \item Mention that data gets split into run and subruns.
    \end{itemize}
    [1 page]
    \section{Calibrations and Detector Modelling}
    \subsection{Detector Monitoring}
    \begin{itemize}
        \item Detector's state is continuously monitored via a number of systems for data quality purposes, including a human detector `shifter'. This includes the alarm systems for the electronics and slow controls, and `nearline' monitoring of the detector status.
        \item CHS and CSS ensure only ``good'' channels used in any analysis.
    \end{itemize}
    [1 page]
    \subsection{Electronic and PMT Calibrations}
    \begin{itemize}
        \item First main set of calibrations are the ECAs and PCAs. These help us convert raw electronic signal information from PMTs into `calibrated' hit times and charges. PCAs performed via TELLIE and the Laserball.
        \item These initial calibrations allow for the first two passes of data processing, resulting in a `RATDS' file format used in optical calibrations.
        \item No need for many details here; this is not a thesis on this topic!
    \end{itemize}
    [1 page]
    \subsection{Energy and Optical Calibrations}
    \begin{itemize}
        \item After ECAs and PCAs, further calibration is necessary to accurately model the optical properties of the detector.
        \item Describe briefly the function of the Laserball, SMELLIE, and AMELLIE sources in optical calibration (can reference the water-phase optical calibration paper). Details of how each analysis works is obviously not needed here, especially not for SMELLIE as we have 3 chapters to go over that!
        \item AmBe and N16 sources provide further information for calibration
        \item In-situ backgrounds used for more calibration sources, especially for determining the light yield of the scintillator, and Birks' constant.
    \end{itemize}
    [2 pages]
    \subsection{Event Reconstruction}
    \begin{itemize}
        \item Once detector is calibrated, event reconstruction becomes possible. Describe main assumption of a SNO+ event: we assume a single-site electron event.
        \item Briefly describe the basics of how energy, position, and time reconstruction works, as this is needed for the solar analysis chapter. Mention existence of direction fitting in scintillator!
        \item `Physics' runs have their events reconstructed in a third pass of data processing, resulting in a `fully-processed' RATDS file as well as a simplified NTUPLE file used in high-level physics analyses (such as my solar analysis).
    \end{itemize}
    [2 pages]
    \subsection{Event Simulation}
    \begin{itemize}
        \item Briefly describe overview of RAT software: not only provides the software by which the above data processing occurs, but allows for a GEANT4-based simulation of events in the detector. This simulation includes all parts of the physics described in Section~\ref{sec:event_journey}, including particle and nuclear physics interactions, optical photon creation and propagation, signal generation and event building based on simulated triggers of the electronics. These simulated events then get processed in the same manner as actual data is.
    \end{itemize}
    [1 page]
    [24 PAGES TOTAL]
}